\documentclass[]{article}

\usepackage{color}
\usepackage{amsmath}
\usepackage{graphicx}
\graphicspath{{./figure/}}
\definecolor{gray}{RGB}{216,216,216}

% Title Page
\title{}
\author{Miao Li}


\begin{document}
\maketitle

\begin{abstract}
\end{abstract}

\section{Introduction}
In this report, we are addressing the problem of variable stiffness control for robust grasping. As shown in our previous work (\textcolor{red}{IROS}), variable stiffness control can effectively stabilize the grasp under disturbance. However, the stability of such controller has never been addressed before. In this work, we are trying to address this problem starting from the simple test examples,see Fig.~\ref{fig::system}.
To this end, first, a detailed formulation for the dynamics of the hand-object is derived, which taken rolling constraints and the soft fingertip into account Then, a variable stiffness controller is presented to grasp the object stable and the upper bound for the change rate of the stiffness is derived from the proof of stability.
\textcolor{red}{(We follow the same notation as in Arimoto's book, add ref.)}
\begin{figure}[h!]
\centering
\includegraphics[height=7cm]{hand_object_system.pdf}
\caption{The hand-object system in 2D. Each finger has 2 DOFs with soft fingertips.}
\label{fig::system}
\end{figure}

\section{Dynamics}
In this part, we first consider the contact model of the fingertips and the constraints involved in the hand-object system. Then the overall dynamics of the system is formulated. 
\subsection{Contact model of soft fingertip}
\begin{equation}
f=c_1(\Delta r)^2+ c_2\frac{d}{dt}\Delta r
\end{equation}
where $c_1$ and $c_2$ are positive constant parameters which depends on the material of the fingertip.
\subsection{Contact constraints}
\begin{figure}[h!]
\centering
\includegraphics[height=5cm]{hand_object_cons1.pdf}
\caption{The constraint that the fingertip should keep contact with the object surface.}
\label{fig::cons1}
\end{figure}
The fingertip should keep contact with the object surface, as shown in Fig.~\ref{fig::cons1}, which can be expressed as follows:
\begin{equation}
l_1+r_1-\Delta r_1=(x-x_1)\cos\theta-(y-y_1)\sin\theta
\label{eqn::cons_contact1}
\end{equation}
\begin{equation}
l_2+r_2-\Delta r_2=-(x-x_2)\cos\theta+(y-y_2)\sin\theta
\label{eqn::cons_contact2}
\end{equation}
\subsection{Rolling constraints}
The rolling constraints on each fingertip can be represented as:
\begin{equation}
(r_i-\Delta r_i)\frac{d}{dt}\phi_i=-\frac{d}{dt}Y_i,\quad i=1,2
\label{eqn::cons_rolling}
\end{equation}
where $Y_i$ and $\phi_i$ are given by:
\begin{equation}
Y_i=(x_i-x)\sin\theta+(y_i-y)\cos\theta
\end{equation}
\begin{equation}
q_{11}+q_{12}+\phi_1=\pi+\theta
\end{equation}
\begin{equation}
q_{21}+q_{22}+\phi_2=\pi-\theta
\end{equation}
\subsection{Overall Dynamics}
The total kinetic energy for the overall system can be described as follows:
\begin{equation}
K=\sum\limits_{i=1,2}\frac{1}{2}\mathbf{\dot{q}}_i^TH_i\mathbf{\dot{q}}_i+\frac{1}{2}
M(\dot{x}^2+\dot{y}^2)+\frac{1}{2}I\dot{\theta}^2
\end{equation}
where $\mathbf{q}_i=[q_{i1},q_{i2}]^T$.

The total potential energy can be given as:
\begin{equation}
P=\sum\limits_{i=1,2}\int_{0}^{\Delta r_i}c_1\eta^2d\eta
\end{equation}
Then from the Hamilton's principle, we have:
\begin{equation}
\int_{t_0}^{t_1}\left[\delta(K-P)-\sum\limits_{i=1,2}c_2\Delta \dot{r}_i
\frac{\partial\Delta r_i}{\partial X^T}\delta X + \sum\limits_{i=1,2}\lambda_i[(r_i-\Delta r_i)\frac{\partial\phi_i}{\partial X^T} + \frac{\partial Y_i}{\partial X^T}]\delta X +\sum\limits_{i=1,2}\mathbf{u_i}^T\delta \mathbf{q}_i\right]dt=0
\end{equation}
where $X=[\mathbf{q}_1^T,\mathbf{q}_2^T,x,y,\theta]^T$.

Then we have the overall dynamics for the object-hand system as follows:
\begin{equation}
H_i(\mathbf{q_i})\ddot{\mathbf{q}}_i+(\frac{1}{2}\dot{H}_i+S_i)\mathbf{\dot{q}}_i+f_i
\frac{\partial \Delta r_i}{\partial \mathbf{q}_i^T}-\lambda_i[(r_i-\Delta r_i)\frac{\partial\phi_i}{\partial \mathbf{q}_i^T} + \frac{\partial Y_i}{\partial \mathbf{q}_i^T}]=\mathbf{u}_i
\end{equation}
\begin{equation}
M\ddot{x}+\sum\limits_{i=1,2}[f_i\frac{\partial \Delta r_i}{\partial x}-\lambda_i
\frac{\partial Y_i}{\partial x}]=0
\end{equation}
\begin{equation}
M\ddot{y}+\sum\limits_{i=1,2}[f_i\frac{\partial \Delta r_i}{\partial y}-\lambda_i
\frac{\partial Y_i}{\partial y}]=0
\end{equation}
\begin{equation}
I\ddot{\theta}+\sum\limits_{i=1,2}[f_i\frac{\partial \Delta r_i}{\partial \theta}-\lambda_i((r_i-\Delta r_i)\frac{\partial\phi_i}{\partial \theta}+
\frac{\partial Y_i}{\partial \theta})]=0
\end{equation}

With the identities in appendix \ref{sec::app1}, the overall dynamics can be simplified as:
\begin{equation}
H_i(\mathbf{q_i})\ddot{\mathbf{q}}_i+(\frac{1}{2}\dot{H}_i+S_i)\mathbf{\dot{q}}_i-(-1)^if_iJ_i^T\begin{bmatrix}
\cos\theta\\
-\sin\theta
\end{bmatrix}-\lambda_i[(r_i-\Delta r_i)\begin{bmatrix}
-1\\-1
\end{bmatrix} + J_i^T\begin{bmatrix}
\sin\theta\\\cos\theta
\end{bmatrix}=\mathbf{u}_i
\end{equation}

\begin{equation}
M\ddot{x}-(f_1-f_2)\cos\theta+(\lambda_1+\lambda_2)\sin\theta=0
\end{equation}
\begin{equation}
M\ddot{y}+(f_1-f_2)\sin\theta+(\lambda_1+\lambda_2)\cos\theta = 0
\end{equation}
\begin{equation}
I\ddot{\theta}-f_1Y_1+f_2Y_2+l_1\lambda_1-l_2\lambda_2=0
\end{equation}
\section{Variable stiffness control}
Motivated from the analysis of the overall system dynamics, the following control law is adopted for each finger to achieve stable grasp:
\begin{equation}
u_i=-D_i\mathbf{q}_i+kJ_i^T(\mathbf{x}_i-\mathbf{x}_d)
\end{equation}
\begin{align}
&\mathbf{x}_i=\begin{bmatrix}
x_i\\
y_i
\end{bmatrix}
&\mathbf{x}_d=\frac{1}{2}\begin{bmatrix}
x_1+x_2\\
y_1+y_2
\end{bmatrix}
\end{align}
where $D_i$ is a diagonal positive definite matrix representing the damping gain. $k$ represents the \emph{variable} stiffness for each fingertip.
\section{Stability Proof}
Taking the sum of inner product of (15) with $\mathbf{\dot{q_i}},i=1,2$, (16) with $\dot{x}$, (17) with $\dot{y}$, (18) with $\dot{\theta}$, we have:
\begin{equation}
\frac{d}{dt}E=-\sum\limits_{i=1,2}\{\mathbf{\dot{q}}_i^TD_i\mathbf{\dot{q}}_i+c_2\Delta \dot{r}_i^2\}+\frac{\dot{k}}{4}{(\mathbf{x}_1-\mathbf{x}_2)^T(\mathbf{x}_1-\mathbf{x}_2)}
\end{equation}
\begin{equation}
E=K+V+P
\end{equation}
\begin{equation}
K=\sum\limits_{i=1,2}\frac{1}{2}\mathbf{\dot{q}}_i^TH_i\mathbf{\dot{q}}_i+\frac{1}{2}
M(\dot{x}^2+\dot{y}^2)+\frac{1}{2}I\dot{\theta}^2
\end{equation}
\begin{equation}
P=\sum\limits_{i=1,2}\int_{0}^{\Delta r_i}c_1\eta^2d\eta
\end{equation}
\begin{equation}
V=\frac{k}{4}(\mathbf{x}_1-\mathbf{x}_2)^T(\mathbf{x}_1-\mathbf{x}_2)
\end{equation}
\textcolor{red}{As proved in Arimoto's book, the closed-loop dynamics is asymptotically stable if:}
\begin{equation}
\frac{d}{dt}E<0
\end{equation}•
which leads to
\begin{equation}
\dot{k}<\frac{\sum\limits_{i=1,2}\{\mathbf{\dot{q}}_i^TD_i\mathbf{\dot{q}}_i+c_2\Delta \dot{r}_i^2\}}{(\mathbf{x}_1-\mathbf{x}_2)^T(\mathbf{x}_1-\mathbf{x}_2)}
\end{equation}
From this, we can conclude the following comments \textcolor{red}{Is this conclusion correct? Any example?}:
\begin{itemize}
\item If the object is softer, $c_2$ is bigger, we can change the stiffness faster.
\item If the object is smaller, $(\mathbf{x}_1-\mathbf{x}_2)^T(\mathbf{x}_1-\mathbf{x}_2)$, we can change the stiffness faster
\end{itemize}•

\section{appendix}
\appendix
\section{Identities}
\label{sec::app1}
From equation (2)(3)(5)(6)(7), we have the following identities:

\colorbox{gray}{
\begin{minipage}{10cm}
 \begin{align*}
 &\frac{\partial \Delta r_1}{\partial \mathbf{q}_1^T}=J_{1}^T\left[\cos\theta,-\sin\theta\right]^T 
 &\frac{\partial \Delta r_2}{\partial \mathbf{q}_2^T}=-J_{2}^T\left[\cos\theta,-\sin\theta\right]^T\nonumber\\
 &\frac{\partial \phi_1}{\partial \mathbf{q}_1^T}=[-1,-1]^T
 &\frac{\partial \phi_2}{\partial \mathbf{q}_2^T}=[-1,-1]^T\nonumber\\
 &\frac{\partial Y_1}{\partial\mathbf{q}_1^T}=J_1^T[\sin\theta,\cos\theta]^T
 &\frac{\partial Y_2}{\partial \mathbf{q}_2^T}=J_2^T[\sin\theta,\cos\theta]^T\nonumber\\
 &\frac{\partial \Delta r_1}{\partial x}=-\cos\theta
 &\frac{\partial \Delta r_2}{\partial x}=\cos\theta \nonumber\\
 &\frac{\partial Y_1}{\partial x}=-\sin\theta 
 &\frac{\partial Y_2}{\partial x}=-\sin\theta \nonumber\\
 &\frac{\partial \Delta r_1}{\partial y}=\sin\theta&\frac{\partial \Delta r_2}{\partial y}=-\sin\theta \nonumber\\
  &\frac{\partial Y_1}{\partial y}=-\cos\theta &\frac{\partial Y_2}{\partial y}=-\cos\theta \nonumber\\
  &\frac{\partial \Delta r_1}{\partial \theta}=-Y_1&\frac{\partial \Delta r_2}{\partial \theta}=Y_2\nonumber\\
  &\frac{\partial \phi_1}{\partial \theta}=1&\frac{\partial \phi_2}{\partial \theta}=-1\nonumber\\
  &\frac{\partial Y_1}{\partial \theta}=(x_1-x)\cos\theta-(y_1-y)\sin\theta&\\
  &\frac{\partial Y_2}{\partial \theta}=(x_2-x)\cos\theta-(y_2-y)\sin\theta&\nonumber
 \end{align*}
\end{minipage}}
\end{document}          
